
\documentclass[12pt]{article}
\usepackage[utf8]{inputenc}
\usepackage[T1]{fontenc}
\usepackage[spanish]{babel}
\usepackage{amsmath,amssymb,mathtools,bm}
\usepackage[a4paper,margin=2.5cm]{geometry}

% === Notación ===
\newcommand{\e}{\mathbf e}   % vectores de la base (primal)
\newcommand{\Tr}{\operatorname{Tr}}
\newcommand{\I}{\mathrm{i}} % i imaginaria

\begin{document}

\begin{center}
\Large\textbf{Bases recíprocas en $\mathbb{R}^3$ e Identidad de Pauli}\\[4pt]
\large Incisos (b)--(d) en $\mathbb{R}^3$ y Punto 7 (Base dual de Pauli)
\end{center}
\bigskip

\section*{Preliminares}
Sea $\{\e_1,\e_2,\e_3\}$ una base (no necesariamente ortonormal) de $\mathbb R^3$ y 
$\{\e^{\,1},\e^{\,2},\e^{\,3}\}$ su base recíproca, definida por
\[
\e^{\,i}\!\cdot \e_j=\delta^{i}{}_{j}\qquad (i,j=1,2,3).
\]
Definimos los volúmenes orientados
\[
V:=\e_1\!\cdot(\e_2\times \e_3), 
\qquad 
\tilde V:=\e^{\,1}\!\cdot(\e^{\,2}\times \e^{\,3}).
\]
En 3D (inciso a):
\[
\e^{\,1}=\frac{\e_2\times \e_3}{V},\qquad
\e^{\,2}=\frac{\e_3\times \e_1}{V},\qquad
\e^{\,3}=\frac{\e_1\times \e_2}{V},\qquad
V=\e_1\!\cdot(\e_2\times \e_3)\neq 0.
\]

\section*{Inciso (b): demostrar que $V\,\tilde V=1$}

\subsection*{Paso 1: calcular $\e^{\,2}\times \e^{\,3}$}
Por bilinealidad del producto cruz,
\[
\e^{\,2}\times \e^{\,3}
=\Big(\frac{\e_3\times \e_1}{V}\Big)\times
 \Big(\frac{\e_1\times \e_2}{V}\Big)
=\frac{1}{V^2}\,\big[(\e_3\times \e_1)\times (\e_1\times \e_2)\big].
\]
Usamos la identidad $x\times(y\times z)=y(x\!\cdot z)-z(x\!\cdot y)$ con 
$x=\e_3\times \e_1$, $y=\e_1$, $z=\e_2$:
\[
(\e_3\times \e_1)\times (\e_1\times \e_2)
= \e_1\big((\e_3\times \e_1)\!\cdot \e_2\big)
     - \e_2\big((\e_3\times \e_1)\!\cdot \e_1\big).
\]
El segundo término es $0$ porque $(\e_3\times \e_1)\perp \e_1$. 
En el primero, por ciclicidad del triple escalar,
\[
(\e_3\times \e_1)\!\cdot \e_2
= \e_1\!\cdot(\e_2\times \e_3)=V.
\]
Por tanto,
\[
(\e_3\times \e_1)\times (\e_1\times \e_2)=V\,\e_1
\quad\Rightarrow\quad
\boxed{\ \e^{\,2}\times \e^{\,3}=\dfrac{\e_1}{V}\ }.
\]

\subsection*{Paso 2: calcular $\tilde V$ y concluir}
Entonces
\[
\tilde V
= \e^{\,1}\!\cdot(\e^{\,2}\times \e^{\,3})
= \e^{\,1}\!\cdot\Big(\frac{\e_1}{V}\Big)
= \frac{1}{V}\,(\e^{\,1}\!\cdot \e_1)
= \frac{1}{V},
\]
ya que $\e^{\,1}\!\cdot \e_1=1$ por definición de base recíproca. 
Finalmente,
\[
\boxed{\,V\,\tilde V=V\cdot \frac{1}{V}=1\,}.
\]

\bigskip
\noindent\textbf{Observación.} Si la base $\{\e_i\}$ es dextrógira, entonces $V>0$ y también $\tilde V>0$.

\section*{Inciso (c): hallar $a$ tal que $a\cdot \e^{\,i}=1$ ($i=1,2,3$) y demostrar unicidad}

\subsection*{Paso 1: expandir $a$ en la base $\{\e_j\}$}
Escribimos
\[
a=a^{j}\,\e_j \quad (\text{convención de suma sobre $j$}).
\]
Por bilinealidad del producto punto,
\[
a\cdot \e^{\,i}=(a^{j}\e_j)\cdot \e^{\,i}=a^{j}\,(\e_j\cdot \e^{\,i})
=a^{j}\,\delta^{i}{}_{j}=a^{i}.
\]
\emph{Conclusión:} para todo $i$, se cumple
\[
a\cdot \e^{\,i}=a^{i}.
\]

\subsection*{Paso 2: imponer la condición $a\cdot \e^{\,i}=1$}
Del paso anterior,
\[
a^{i}=1\quad\text{para } i=1,2,3.
\]
Por lo tanto,
\[
\boxed{\,a=\e_1+\e_2+\e_3\,}.
\]

\subsection*{Verificación}
Usando $\e^{\,i}\!\cdot \e_j=\delta^{i}{}_{j}$,
\[
a\cdot \e^{\,1}=(\e_1+\e_2+\e_3)\cdot \e^{\,1}=1+0+0=1,
\]
y análogamente $a\cdot \e^{\,2}=1$ y $a\cdot \e^{\,3}=1$.

\subsection*{Unicidad}
Si $a$ y $b$ satisfacen $a\cdot \e^{\,i}=b\cdot \e^{\,i}=1$ ($i=1,2,3$), entonces
\[
(a-b)\cdot \e^{\,i}=0\quad (i=1,2,3).
\]
Pero $(a-b)\cdot \e^{\,i}=(a-b)^{i}$, de modo que 
\[
(a-b)^{1}=(a-b)^{2}=(a-b)^{3}=0 \;\Rightarrow\; a-b=0 \;\Rightarrow\; a=b.
\]
\[
\boxed{\text{El vector } a \text{ es único.}}
\]

\section*{Inciso (d). Base recíproca y componentes de $a$ en un sistema oblicuo}

Sean
\[
\e_1=(4,2,1),\qquad 
\e_2=(3,3,0),\qquad 
\e_3=(0,0,2),
\]
una base \emph{dextrógira} (no necesariamente ortogonal). Sea además $\mathbf a=(1,2,3)$.

\subsection*{I) Cálculo de la base recíproca $\{\e^{\,i}\}$}

La base recíproca $\{\e^{\,i}\}$ se define por
\[
\e^{\,i}\!\cdot \e_j=\delta^i{}_j\qquad (i,j=1,2,3).
\]
Para una base dextrógira, una construcción práctica es
\[
\boxed{\quad 
\e^{\,i}=\dfrac{\e_j\times \e_k}{V},
\qquad 
V:=\e_1\cdot(\e_2\times \e_3),\quad (i,j,k)\ \text{cíclico}.
\quad}
\]

\paragraph{Volumen escalar.}
\[
\e_2\times \e_3=
\begin{vmatrix}
\mathbf i& \mathbf j& \mathbf k\\
3&3&0\\
0&0&2
\end{vmatrix}=(6,-6,0),
\qquad
V=\e_1\cdot(\e_2\times \e_3)=(4,2,1)\cdot(6,-6,0)=12.
\]

\paragraph{Vectores recíprocos.}
Con permutaciones cíclicas:
\[
\begin{aligned}
\e^{\,1}&=\frac{\e_2\times \e_3}{V}
=\frac{(6,-6,0)}{12}=\Big(\tfrac12,-\tfrac12,0\Big),\\[2mm]
\e^{\,2}&=\frac{\e_3\times \e_1}{V}
=\frac{(-4,8,0)}{12}=\Big(-\tfrac13,\tfrac23,0\Big),\\[2mm]
\e^{\,3}&=\frac{\e_1\times \e_2}{V}
=\frac{(-3,3,6)}{12}=\Big(-\tfrac14,\tfrac14,\tfrac12\Big).
\end{aligned}
\]
Se verifica que $\e^{\,i}\!\cdot \e_j=\delta^i{}_j$.

\subsection*{II) Componentes covariantes y contravariantes de $\mathbf a=(1,2,3)$}

En una base oblicua,
\[
\mathbf a=a^{\,i}\e_i=a_i\,\e^{\,i},
\qquad
a^{\,i}=\mathbf a\cdot \e^{\,i},\quad
a_i=\mathbf a\cdot \e_i .
\]

\paragraph{Componentes contravariantes $a^{\,i}=\mathbf a\cdot \e^{\,i}$.}
\[
\begin{aligned}
a^{\,1}&=(1,2,3)\cdot\Big(\tfrac12,-\tfrac12,0\Big)=\tfrac12-1=-\tfrac12,\\
a^{\,2}&=(1,2,3)\cdot\Big(-\tfrac13,\tfrac23,0\Big)=-\tfrac13+\tfrac43=1,\\
a^{\,3}&=(1,2,3)\cdot\Big(-\tfrac14,\tfrac14,\tfrac12\Big)=-\tfrac14+\tfrac12+\tfrac32=\tfrac74.
\end{aligned}
\]

\paragraph{Componentes covariantes $a_i=\mathbf a\cdot \e_i$.}
\[
\begin{aligned}
a_1&=(1,2,3)\cdot(4,2,1)=4+4+3=11,\\
a_2&=(1,2,3)\cdot(3,3,0)=3+6+0=9,\\
a_3&=(1,2,3)\cdot(0,0,2)=0+0+6=6.
\end{aligned}
\]

\paragraph{Comprobación.}
Ambas descomposiciones recuperan el mismo vector:
\[
\mathbf a
= a^{\,i}\e_i=(-\tfrac12)\e_1+1\cdot \e_2+\tfrac74\,\e_3
= a_i\,\e^{\,i}=11\,\e^{\,1}+9\,\e^{\,2}+6\,\e^{\,3}
=(1,2,3).
\]

\subsection*{Observación (útil para el producto vectorial en base oblicua)}
De la identidad
\(
\e_i\times \e_j=V\,\varepsilon_{ijk}\,\e^{\,k}
\)
se deduce, para $\mathbf a=a^{\,i}\e_i$ y $\mathbf b=b^{\,j}\e_j$,
\[
\boxed{\quad
\mathbf a\times \mathbf b
=V\,\varepsilon_{ijk}\,a^{\,i}b^{\,j}\,\e^{\,k},
\qquad V=\e_1\cdot(\e_2\times \e_3)=12.
\quad}
\]

% ----------------------------------------------------------------------
\section*{Punto 7: Base dual asociada a la base de Pauli y 1-forma asociada}

Sea $H_2$ el espacio \emph{real} de matrices herm\'iticas $2\times 2$, con el producto interno
\[
\langle A,B\rangle=\Tr(A^\dagger B).
\]
Como $A,B$ son herm\'iticas ($A^\dagger=A$, $B^\dagger=B$), aqu\'i $\langle A,B\rangle=\Tr(AB)$.
Tomamos la base de Pauli
\[
\sigma_0=\begin{pmatrix}1&0\\[2pt]0&1\end{pmatrix},\quad
\sigma_1=\begin{pmatrix}0&1\\[2pt]1&0\end{pmatrix},\quad
\sigma_2=\begin{pmatrix}0&-\I\\[2pt]\I&0\end{pmatrix},\quad
\sigma_3=\begin{pmatrix}1&0\\[2pt]0&-1\end{pmatrix}.
\]

\subsection*{1. Matriz de Gram de la base de Pauli}
Definimos la matriz de Gram $G_{\mu\nu}=\langle\sigma_\mu,\sigma_\nu\rangle=\Tr(\sigma_\mu\sigma_\nu)$.
Mostremos las identidades clave:
\[
\Tr(\sigma_0\sigma_0)=2,\qquad
\Tr(\sigma_0\sigma_k)=0\ (k=1,2,3),\qquad
\Tr(\sigma_i\sigma_j)=2\,\delta_{ij}\ (i,j=1,2,3).
\]
\paragraph{Esbozo de verificación.}
(i) $\Tr(\sigma_0\sigma_0)=\Tr(I_2)=2$. \ 
(ii) $\Tr(\sigma_0\sigma_k)=\Tr(\sigma_k)=0$ porque cada $\sigma_k$ (con $k=1,2,3$) es \emph{traceless}.\ 
(iii) Para $i=j$, $\sigma_i^2=I_2\Rightarrow \Tr(\sigma_i\sigma_i)=2$; 
para $i\neq j$, $\sigma_i\sigma_j=\I\varepsilon_{ijk}\sigma_k$ cuya traza es $0$. 
En conjunto, $G_{\mu\nu}=2\,\delta_{\mu\nu}$, esto es
\[
G=2\,I_4, \qquad G^{-1}=\tfrac12\,I_4.
\]
La base de Pauli es ortogonal (no ortonormal): $\|\sigma_\mu\|^2=\Tr(\sigma_\mu^2)=2$.

\subsection*{2. Base dual}
Por definici\'on, la base dual $\{\sigma^\mu\}$ satisface
$\langle\sigma^\mu,\sigma_\nu\rangle=\delta^\mu{}_\nu$.
Si escribimos
\(
\sigma^\mu=\sum_{\rho=0}^3 c^{\mu}{}_{\rho}\,\sigma_\rho
\)
e imponemos la condici\'on dual, en notaci\'on matricial se obtiene $C\,G=I_4$,
de modo que $C=G^{-1}$ y finalmente
\[
\boxed{\ \sigma^\mu=\tfrac12\,\sigma_\mu\qquad(\mu=0,1,2,3)\ }.
\]
Verificaci\'on directa:
\(
\langle\tfrac12\sigma_\mu,\sigma_\nu\rangle
=\tfrac12\,\Tr(\sigma_\mu\sigma_\nu)
=\tfrac12\cdot 2\,\delta_{\mu\nu}
=\delta^\mu{}_\nu.
\)

\subsection*{3. 1-forma asociada y componentes}
Todo $A\in H_2$ se expande de forma \'unica como
\[
A=a^\mu\,\sigma_\mu\qquad(\mu=0,1,2,3).
\]
Los \emph{coeficientes contravariantes} se recuperan aplicando la base dual:
\[
\boxed{\ a^\mu=\langle\sigma^\mu, A\rangle=\tfrac12\,\Tr(\sigma_\mu A)\ }.
\]
La 1-forma asociada a $A$ (también llamada “bemol” de $A$) es
\[
A^\flat(\cdot)=\langle A,\cdot\rangle=\Tr(A\,\cdot),
\]
y sus \emph{componentes covariantes} respecto a la co-base
$\varepsilon^\mu(\,\cdot\,):=\langle\sigma^\mu,\,\cdot\,\rangle
=\tfrac12\,\Tr(\sigma_\mu\,\cdot\,)$ son
\[
a_\mu:=\langle A,\sigma_\mu\rangle=\Tr(A\sigma_\mu)
=G_{\mu\nu}\,a^\nu=2\,a^\mu.
\]
En resumen,
\[
\boxed{\ a_\mu=2\,a^\mu,\qquad
A^\flat=\sum_{\mu=0}^3 a_\mu\,\varepsilon^\mu
=\sum_{\mu=0}^3 (2a^\mu)\,\varepsilon^\mu\ }.
\]

\subsection*{4. Parámetros reales explícitos (opcional)}
Si
\[
A=\alpha\,\sigma_0+\gamma\,\sigma_1+\delta\,\sigma_2+\beta\,\sigma_3
=
\begin{pmatrix}
\alpha+\beta & \gamma-\I\delta\\[2pt]
\gamma+\I\delta & \alpha-\beta
\end{pmatrix},
\]
entonces
\[
a^0=\tfrac12\Tr(\sigma_0A)=\alpha,\quad
a^1=\tfrac12\Tr(\sigma_1A)=\gamma,\quad
a^2=\tfrac12\Tr(\sigma_2A)=\delta,\quad
a^3=\tfrac12\Tr(\sigma_3A)=\beta,
\]
y $a_\mu=2a^\mu$. En particular, $A^\flat(X)=\Tr(AX)$ para todo $X\in H_2$.

\subsection*{Anexo: comprobaciones matriciales básicas}
Para completar, verificamos dos productos cruzados (las demás son análogas):
\begin{align*}
\sigma_2\sigma_3&=
\begin{pmatrix}0&-\I\\ \I&0\end{pmatrix}
\begin{pmatrix}1&0\\ 0&-1\end{pmatrix}
=
\begin{pmatrix}0&\I\\ \I&0\end{pmatrix}
=\I\,\sigma_1,
\\[2pt]
\sigma_3\sigma_1&=
\begin{pmatrix}1&0\\ 0&-1\end{pmatrix}
\begin{pmatrix}0&1\\ 1&0\end{pmatrix}
=
\begin{pmatrix}0&1\\ -1&0\end{pmatrix}
=-\I\,\sigma_2.
\end{align*}
En ambos casos, la traza es $0$ porque $\Tr(\sigma_k)=0$ para $k=1,2,3$.

\end{document}
